\renewcommand\thesection{II}
\section{Dynamique du Point Matériel}

\begin{multicols*}{2}
    \subsection{Force}
    Une force est une grandeur caractérisée par une intensité, une direction, et un sens, qui suit les lois d'addition vectorielles. Elle est exprimée en Newtons ($N$).
    
    Une force peut être exercée sur un objet:
    \begin{itemize}
        \item Par contact (ex. Force de poussée par quelqu'un poussant un wagonnet).
        \item Indirectement (ex. Force de répulsion entre deux charges de même signe).
    \end{itemize}
    
    \paragraph{Dynamomètre}
    Outil de mesure de l'intensité d'une force.
    
    \subsection{Lois de Newton}
    \subsubsection{1e Loi de Newton}
    Tout corps reste immobile ou conserve un MRU aussi longtemps qu'aucune force extérieure vient modifier son état.
    \[ \vec F = 0 \Leftrightarrow v(t) = v_0 \]
    
    \subsubsection{2e Loi de Newton}
    L'accélération dépend de la force et de la masse.
    \[ \vec F = m \cdot \vec a \]
    
    \paragraph{Newton} 
    Force nécessaire pour accélérer un objet de $1kg$ de $1m\cdot s^{-2}$
    \[ 1N = \frac{1 kg \cdot m}{s^2}\]
    
    \subsubsection{3e Loi de Newton}
    Chaque fois qu'un objet exerce une force sur un second objet, ce dernier exerce sur le premier une force égale et opposée.
    \[ \vec F_{12} = - \vec F_{21} \]
    
    \subsubsection{Domaine de Validité}
    Les Lois de Newton sont vérifiées par l'expérience à deux conditions:
    \begin{enumerate}
        \item Appliquées dans un référentiel inertiel (MRU par rapport aux galaxies lointaines)
        \item Vitesses des mobiles petites par rapport à la vitesse de la lumière ($300 000 km\cdot s^{-1}$)
    \end{enumerate}
    
    \subsection{Force Gravitationelle}
    Il existe entre deux objets ponctuels de masse $m_1$ et $m_2$ distants de $r_{12}$ une force d'attraction $\vec F_G$ dont le module est donné par:
    \[ F_G = G\frac{m_1\cdot m_2}{r_{12}^2} \]
    Avec $G$ la constante de Gravitation:
    \[G = 6,67384 \cdot 10^{-11} N \cdot m^2 \cdot kg^{-2} \]
    
    \subsection{Poids}
    Un objet ponctuel de masse m à la surface de la Terre subit une force d'attraction gravitationnelle de la part de chaque molécule constituant la Terre.
    
    \[ ||\vec P|| = m \sum_i G \frac{m_i}{r_i^2} \hat\imath \qquad \Rightarrow \qquad \vec P = m \cdot \vec g \]
    
    Avec $g$ L'accélération de la pesanteur Terrestre:
    
    \[ g = \left| \sum_i G \frac{m_i}{r_i^2} \hat\imath \right| = 9,81 m\cdot s^{-2} \qquad \text{à Bruxelles}\]
    
\end{multicols*}
\renewcommand\thesection{IX}
\section{Appareils de Mesures Électriques}

\begin{multicols*}{2}
    \subsection{Voltmètre}

    Mesure la différence de potentiel (Volts) entre deux points du circuit. Il s'attache en parallèle: ici aux points $A$ et $B$.
    
    \begin{center}
        \begin{circuitikz}
            \draw (0,0) to [battery1, l_ =$\xi$] (0,2) to [R = $R_1$,-*] (2,2) node[above] {$A$} to [R = $R_2$,-*] (2,0) node[below] {$B$} to [R= $R_3$] (0,0);
            \draw (2,2) to (4, 2) to [rmeter, t=V] (4,0) to (2,0);
        \end{circuitikz}
    \end{center}
    
    \subsection{Ampèremètre}
    
    Mesure l'intensité du courant (Ampères) en un point du circuit. Il s'attache en série: ici on mesure l'intensité du courant passant dans la résistance $R_3$.
    
    \begin{center}
        \begin{circuitikz}
            \draw (0,0) to [battery1, l_ =$\xi$] (0,2) to [R = $R_1$] (2,2) to [R = $R_2$] (2,0) to (0,0);
            \draw (2,2) to [rmeter, t=A] (4, 2) to [R = $R_3$] (4,0) to (2,0);
        \end{circuitikz}
    \end{center}
    
    \subsection{Ohmmètre}
    
    Mesure la valeur d'une résistance. Il s'attache aux deux extrémités d'un circuit composé de résistances non alimenté. Ci dessous on mesure la résistance de $R_3$.
    
    \begin{center}
        \begin{circuitikz}
            \draw (0,0) to [battery1, l_ =$\xi$] (0,2) to [R = $R_1$] (2,2) to [R = $R_2$] (2,0) to (0,0);
            \draw (3.5,2) to [R = $R_3$] (3.5, 0) to (2,0);
            \draw (3.5,2) to  (5, 2) to [rmeter, t=$\Omega$] (5,0) to (3.5,0);
        \end{circuitikz}
    \end{center}
    
    \subsection{Multimètre}
    
    Appareil regroupant les différents appareils décrits ci-dessus. Il peut être réglé pour être utilisé soit comme voltmètre, soit comme ampèremètre, soit comme ohmmètre.
    
    \paragraph{Galvanomètre}
    Un Galvanomètre fait dévier une aiguille sur un cadran de manière proportionnelle au courant qui le traverse. Un multimètre est composé de galvanomètres pour pouvoir indiquer les valeurs sur son cadran.
    
    \paragraph{Voltmètre}
    
    \begin{center}
        \begin{circuitikz}
            \draw (0, 0) to (1, 0) to [R = $R$] (3, 0) to [R = $r$] (4.5, 0) to [rmeter, t=$G$] (6, 0) to (7,0);
        \end{circuitikz}
    \end{center}
    
    \paragraph{Ampèremètre}
    \begin{center}
        \begin{circuitikz}
            \draw (-0.5, 0) to (1, 0) to [R = $R$] (4,0) to (5.5,0);
            \draw (1, 0) to (1, 1) to [R = $r$] (2.5, 1) to [rmeter, t=$G$] (4, 1) to (4, 0);
        \end{circuitikz}
    \end{center}
    
    \paragraph{Ohmmètre}
    
    \begin{center}
        \begin{circuitikz}
            \draw (0.5, 0) to (1, 0) to [battery1, l_=$V$] (3, 0) to [R = $r$] (4.5, 0) to [rmeter, t=$G$] (6, 0) to (7,0);
        \end{circuitikz}
    \end{center}
    
    \subsection{Oscilloscope}
    
    Permet l'étude des tensions alternatives en observant la forme de la variation dans le temps.
    
    \begin{center}
        \begin{circuitikz}
            \draw (0,0) to [oscope] (3,0);
        \end{circuitikz}
    \end{center}
    
\end{multicols*}